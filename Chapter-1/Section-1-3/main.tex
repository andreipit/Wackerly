\documentclass[12pt]{article}
\usepackage{fullpage}
\usepackage{amsthm, amsmath, amssymb, mathrsfs, mathtools, booktabs, array, graphicx, subcaption, caption, enumitem, listings, setspace, upgreek, float}

\newcommand{\Z}{\mathcal{Z}}
\newcommand{\R}{\mathbb{R}}
\newcommand{\bx}{\textbf{x}}

\graphicspath{ {figures/} }


\addtolength{\oddsidemargin}{-.875in}
\addtolength{\evensidemargin}{-.875in}
\addtolength{\textwidth}{1.75in}

\addtolength{\topmargin}{-.875in}
\addtolength{\textheight}{1.75in}

\title{Wackerly Section 1.3 Problems}
\author{John Nguyen}

\begin{document}
\maketitle

\subsection*{Problem 1.9}
    \begin{enumerate}[label=(\alph*).]
        \item By the Empirical Rule, about 0.68.
        
        \item By the Empirical Rule, about 0.95.
        
        \item According to the Empirical Rule, about 0.765.
        
        \item According to the Empirical Rule, almost 0.0.
        
    \end{enumerate}

\subsection*{Problem 1.10}
    \begin{enumerate}[label=(\alph*).]
        \item 14 - 17 = -3
        
        \item Assuming time spent online using the Internet is approximately normally distributed, about 0.16 of people spend less than -3 hours online per year.
        
        \item No, because it is impossible to spend less than 0 hours online per year. Thus the normal distribution assumption is false.
        
    \end{enumerate}
    
\subsection*{Problem 1.11}
    \begin{align*}
        s^2 &= \frac{1}{n-1} \sum_{i = 1}^n (y_i - \bar{y})^2 \\
        &= \frac{1}{n-1} \sum_{i = 1}^n (y_i^2 - 2y_i \bar{y} + \bar{y}^2) \\
    \end{align*}
    
    Using (c),
    
    \begin{align*}
        \frac{1}{n-1} \sum_{i = 1}^n (y_i^2 - 2y_i \bar{y} + \bar{y}^2) &= \frac{1}{n-1} [\sum_{i = 1}^n y_i^2 + \sum_{i = 1}^n (-2y_i\bar{y} + \bar{y}^2)] \\
        &= \frac{1}{n-1} [\sum_{i = 1}^n y_i^2 + \sum_{i = 1}^n (-2y_i\bar{y}) + \sum_{i = 1}^n \bar{y}^2]
    \end{align*}
    
    Using (a),
    
    \begin{align*}
        \frac{1}{n-1} [\sum_{i = 1}^n y_i^2 + \sum_{i = 1}^n (-2y_i\bar{y}) + \sum_{i = 1}^n \bar{y}^2] &= \frac{1}{n-1} [\sum_{i = 1}^n y_i^2 + \sum_{i = 1}^n (-2y_i \bar{y}) + n \bar{y}^2]
    \end{align*}
    
    Using (b),
    
    \begin{align*}
        \frac{1}{n-1} [\sum_{i = 1}^n y_i^2 + \sum_{i = 1}^n (-2y_i \bar{y}) + n \bar{y}^2] &= \frac{1}{n-1} [\sum_{i = 1}^n y_i^2 + \bar{y} \sum_{i = 1}^n (-2y_i) + n \bar{y}^2 ] \\
        &= \frac{1}{n-1} \left[\sum_{i = 1}^n y_i^2 + \left(\frac{1}{n} \sum_{j = 1}^n y_j \right) \sum_{i = 1}^n (-2y_i) + n \left(\frac{1}{n} \sum_{i = 1}^n y_i \right)^2 \right]
    \end{align*}
    
    Consider the term $\left(\frac{1}{n} \sum_{j = 1}^n y_j \right) \sum_{i = 1}^n (-2y_i)$. Notice that when j = i:
    
    \[ \left(\frac{1}{n} \sum_{j = 1}^n y_j \right) \left( \sum_{i = 1}^n (-2y_i) \right) = \frac{1}{n} \sum_{i = 1}^n -2y_i^2 \]
    
    Meanwhile, if $i \neq j$:
    
    \[ \left(\frac{1}{n} \sum_{j = 1}^n y_j \right) \left( \sum_{i = 1}^n (-2y_i) \right) = \frac{1}{n} \sum_{j = 1}^n \sum_{i = 1}^n (-4y_iy_j) \]
    
    Therefore,
    
    \begin{align*}
        \frac{1}{n-1} \left[\sum_{i = 1}^n y_i^2 + \left(\frac{1}{n} \sum_{j = 1}^n y_j \right) \sum_{i = 1}^n (-2y_i) + n \left(\frac{1}{n} \sum_{i = 1}^n y_i \right)^2 \right] &= \\
        \frac{1}{n-1} \left[ \sum_{i = 1}^n y_i^2 + \frac{1}{n} \sum_{i = 1}^n -2y_i^2 + \frac{1}{n} \sum_{j = 1}^n \sum_{i = 1}^n (4 y_i y_j) + \frac{1}{n} \left( \sum_{i = 1}^n y_i \right)^2 \right]
    \end{align*}
    
    Notice that $\frac{1}{n}\left( \sum_{i = 1}^n y_i \right)^2 = \frac{1}{n} \sum_{i = 1}^n y_i^2 + \frac{1}{n} \sum_{j = 1}^n \sum_{i = 1}^n 2 y_i y_j$. Therefore,
    
    \[ \frac{1}{n-1} \left[ \sum_{i = 1}^n y_i^2 + \frac{1}{n} \sum_{i = 1}^n -2y_i^2 + \frac{1}{n} \sum_{j = 1}^n \sum_{i = 1}^n (4 y_i y_j) + \frac{1}{n} \left( \sum_{i = 1}^n y_i \right)^2 \right] =  \]
    \[ \frac{1}{n-1} \left[ \sum_{i = 1}^n y_i^2 + \frac{1}{n} \sum_{i = 1}^n -2y_i^2 + \frac{1}{n} \sum_{j = 1}^n \sum_{i = 1}^n (-4y_iy_j) + \frac{1}{n} \sum_{i = 1}^n y_i^2 + \frac{1}{n} \sum_{j = 1}^n \sum_{i = 1}^n 2y_iy_j \right] \]
    
    So,
    
    \begin{align*}
        \frac{1}{n-1} \left[ \sum_{i = 1}^n y_i^2 - \frac{1}{n} \sum_{i = 1}^n y_i^2 - \frac{1}{n} \sum_{j = 1}^n \sum_{i = 1}^n 2y_iy_j \right] &= \frac{1}{n - 1} \left[ \sum_{i = 1}^n y_i^2 - \frac{1}{n} \left( \sum_{i = 1}^n y_i^2 + \sum_{j = 1}^n \sum_{i = 1}^n 2 y_i y_j \right) \right] \\
        &= \frac{1}{n-1} \left[ \sum_{i = 1}^n y_i^2 - \frac{1}{n} \left( \sum_{i = 1}^n y_i \right)^2 \right]
    \end{align*}
    

\subsection*{Problem 1.12}
    (see code) \\
    
    s = 1.4667
\subsection*{Problem 1.13}
    (see code) \\
    
    \begin{enumerate}[label=(\alph*).]
        \item $\mu = 9.7911, s^2 = 17.1231, s = 4.1380$
        
        \item In the interval [5.65:, 13.93]. \\
        According to the Empirical Rule, I expect to find 31 \\
        I found 44 samples. \\
        
        
        In the interval [1.52:, 18.07]. \\
        According to the Empirical Rule, I expect to find 43\\
        I found 44 samples.\\
        
        
        In the interval [-2.62:, 22.21].\\
        According to the Empirical Rule, I expect to find 45\\
        I found 44 samples.\\
        
        The results do not follow the Empirical Rule because the Empirical Rule assumes my data obeys an approximately normal distribution.
        This is not the case because there is 1 extreme outlier sample in my data which causes my standard deviation to inflate.
        A more accurate analysis can be done by removing the outlier 35.1 value.
        
    \end{enumerate}

\subsection*{Problem 1.14}
    \begin{enumerate}[label=(\alph*).]
        \item $\mu = 3.2252, s = 3.1029$
        
        \item In the interval [0.12:, 6.33].\\
        According to the Empirical Rule, I expect to find 17\\
        I found 21 samples.\\
        
        
        In the interval [-2.98:, 9.43].\\
        According to the Empirical Rule, I expect to find 24\\
        I found 23 samples.\\
        
        
        In the interval [-6.08:, 12.53].\\
        According to the Empirical Rule, I expect to find 25\\
        I found 25 samples.\\
        
        These results are what I expect.\\
        
    \end{enumerate}
    
\subsection*{Problem 1.15}
    \begin{enumerate}[label=(\alph*).]
        \item $\mu = 4.387, s = 1.8714$
        
        \item In the interval [2.52:, 6.26].\\
        According to the Empirical Rule, I expect to find 27\\
        I found 35 samples.\\
        
        
        In the interval [0.64:, 8.13].\\
        According to the Empirical Rule, I expect to find 38\\
        I found 39 samples.\\
        
        
        In the interval [-1.23:, 10.00].\\
        According to the Empirical Rule, I expect to find 40\\
        I found 39 samples.\\
        
        For lower values of k, I found more samples than expected. This is likely due to the outlier value 11.88 in the data. Removing that sample will cause the data to better fit the Empirical Rule.
        
    \end{enumerate}

\subsection*{Problem 1.16}
    \[ \mu = 4.1949, s = 1.4232 \]
    \noindent
    In the interval [2.77:, 5.62].\\
    According to the Empirical Rule, I expect to find 27 \\
    I found 25 samples.\\
    
    \noindent
    In the interval [1.35:, 7.04].\\
    According to the Empirical Rule, I expect to find 37\\
    I found 36 samples.\\
    
    \noindent
    In the interval [-0.07:, 8.46].\\
    According to the Empirical Rule, I expect to find 39\\
    I found 39 samples.\\
    
    \noindent
    These results almost perfectly follow the Empirical Rule, as expected.
    By removing the outlier, the mean decreased by 0.19 and standard deviation went from 1.85 to 1.42.\\

\subsection*{Problem 1.17}
    Analyzing Problem 1.2.\\
    The standard deviation is 4.09.\\
    The approximation is 7.35.\\
    There is a large difference between the true standard deviation and the approximation because of the 35.1 outlier value. This is consistent with our analysis in Problem 1.13 where we found the Empirical Rule did not fit the data for similar reasons.\\
    
    \noindent
    Analyzing Problem 1.3.\\
    The standard deviation is 3.10.\\
    The approximation is 3.04.\\
    The true standard deviation and the approximation are close, which is consistent with our analysis in Problem 1.14.\\
    
    \noindent
    Analyzing Problem 1.4.\\
    The standard deviation is 1.85.\\
    The approximation is 2.32.\\
    There is a large difference between the standard deviation and approximation because of the outlier sample 11.88. This is consistent with our analysis in Problems 1.15 and 1.16 concerning the Empirical Rule.\\
    
\subsection*{Problem 1.18}
    Using the approximation given in Problem 1.17, the standard deviation of the verbal test should be about 150.0.

\subsection*{Problem 1.19}
    Chlorphorm amounts do not follow a normal distirbution because if it did, the data should be approximated by the Empirical Rule.
    Therefore, about 2.5\% of public water samples should have cholorphorm values between (-125.00, -72.00).
    However, it is physically impossible to record a chlorophorm sample below 0 micrograms per liter. Due to this physical constraint, the chlorophorm samples cannot follow a normal distribution with mean 34 and standard deviation 53.
\subsection*{Problem 1.20}
    Assuming maitenance cost is normally distirbuted with mean 420.00 and standard deivaiton is 30.00, the Empirical Rule implies that the probability of running overbudget with a budget of 450.00 is 16\%. 
    
\subsection*{Problem 1.21}
    Assuming cow weight fed this diet follows a normal distribution with mean 22 and standard deviation 2, the Emprical Rule implies the probability of weight gain exceeding 20 pounds in 82\%.
    Thus the manufacturers claim is supported by these results.

\end{document}